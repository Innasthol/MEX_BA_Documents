\documentclass[12pt,a4paper]{article}
\usepackage[left=3.5cm,right=2.5cm,top=1.5cm,bottom=1.5cm,includeheadfoot]{geometry}
\usepackage[onehalfspacing]{setspace}
\usepackage{ngerman}
\usepackage[latin1]{inputenc}
\usepackage[T1]{fontenc}
\usepackage{graphicx}

\begin{document}
	\begin{titlepage}
	\begin{center}
		\huge \textbf{\textsc{Umsetzung einer objektorientierten Software-Architektur in C++ zur plattformunabh�ngigen Ansteuerung eines Robotermanipulators}} \\  
		\ \\ 
		\ \\
		\LARGE{ 
			\textbf{
				Bachelorarbeit vorgelegt von \\ 
				Willi Penner \\ 
				Matrikelnummer: 1106136 \\
			}
		}
		\ \\
		\ \\
	\end{center}
	
	\begin{flushleft}
		Angefertigt im Studiengang Mechatronik \\
		(Bachelor of Science, B. Sc.) \\
		\ \\
		Am Fachbereich Ingenierwissenschaften und Mathematik der Fachhochschule Bielefeld \\
		\ \\
		\ \\
		Tag der Abgabe: 24.07.2020 \\
		Sommersemester 2020 \\
		\ \\
		Erstpr�fer: Prof. Dr. rer. nat. Martin H�lse \\
		Zweitpr�fer: Prof. Dr. rer. nat. Axel Schneider \\
	\end{flushleft}    
\end{titlepage}
	
\section{Vorwort}
Text

\section{Inhaltsverzeichnis}
Text

\section{Abstract}
Text

\section{Abk�rzungsverzeichnis}
Text

\section{Aufgabenstellung}
Text

\section{Voraussetzungen}
Text

\subsection{MEX-Projekt Beschreibung}
Text

\subsection{Testumgebung (Hardware)}
Text

\section{Umsetzung der Schnittstelle}

\section{Aufbau und Analyse des Robotermanipulators}

\section{Dokumentation und Voraussetzungen f�r die Nutzung der Schnittstelle}

\section{Quellenverzeichnis}
Text

\section{Abbildungsverzeichnis}
Text

\section{Eidesstattliche Versicherung}
Text

\section{Anhang}
Text

\end{document}
